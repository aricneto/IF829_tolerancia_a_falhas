%%%%%%%%%%%%%%%%%%%%%%%%%%%%%%%%%%%%%%%%%
% Lachaise Assignment
% LaTeX Template
% Version 1.0 (26/6/2018)
%
% This template originates from:
% http://www.LaTeXTemplates.com
%
% Authors:
% Marion Lachaise & François Févotte
% Vel (vel@LaTeXTemplates.com)
%
% License:
% CC BY-NC-SA 3.0 (http://creativecommons.org/licenses/by-nc-sa/3.0/)
% 
%%%%%%%%%%%%%%%%%%%%%%%%%%%%%%%%%%%%%%%%%

%----------------------------------------------------------------------------------------
%	PACKAGES AND OTHER DOCUMENT CONFIGURATIONS
%----------------------------------------------------------------------------------------

\documentclass{article}

%%%%%%%%%%%%%%%%%%%%%%%%%%%%%%%%%%%%%%%%%
% Lachaise Assignment
% Structure Specification File
% Version 1.0 (26/6/2018)
%
% This template originates from:
% http://www.LaTeXTemplates.com
%
% Authors:
% Marion Lachaise & François Févotte
% Vel (vel@LaTeXTemplates.com)
%
% License:
% CC BY-NC-SA 3.0 (http://creativecommons.org/licenses/by-nc-sa/3.0/)
% 
%%%%%%%%%%%%%%%%%%%%%%%%%%%%%%%%%%%%%%%%%

%----------------------------------------------------------------------------------------
%	PACKAGES AND OTHER DOCUMENT CONFIGURATIONS
%----------------------------------------------------------------------------------------

\usepackage{amsmath,amsfonts,stmaryrd,amssymb} % Math packages

\usepackage{enumerate} % Custom item numbers for enumerations

\usepackage[ruled]{algorithm2e} % Algorithms

\usepackage[framemethod=tikz]{mdframed} % Allows defining custom boxed/framed environments

\usepackage{listings} % File listings, with syntax highlighting
\lstset{
	basicstyle=\ttfamily, % Typeset listings in monospace font
}

\usepackage{derivative} % support for derivatives
\usepackage{siunitx} % support for SI units

\usepackage{empheq} 
\usepackage{xcolor}
\definecolor{lightgreen}{HTML}{90EE90}

%----------------------------------------------------------------------------------------
%	DOCUMENT MARGINS
%----------------------------------------------------------------------------------------

\usepackage{geometry} % Required for adjusting page dimensions and margins

\geometry{
	paper=a4paper, % Paper size, change to letterpaper for US letter size
	top=2.5cm, % Top margin
	bottom=3cm, % Bottom margin
	left=2.5cm, % Left margin
	right=2.5cm, % Right margin
	headheight=14pt, % Header height
	footskip=1.5cm, % Space from the bottom margin to the baseline of the footer
	headsep=1.2cm, % Space from the top margin to the baseline of the header
	%showframe, % Uncomment to show how the type block is set on the page
}

%----------------------------------------------------------------------------------------
%	FONTS
%----------------------------------------------------------------------------------------

\usepackage[utf8]{inputenc} % Required for inputting international characters
\usepackage[T1]{fontenc} % Output font encoding for international characters

\usepackage{XCharter} % Use the XCharter fonts

%----------------------------------------------------------------------------------------
%	COMMAND LINE ENVIRONMENT
%----------------------------------------------------------------------------------------

% Usage:
% \begin{commandline}
%	\begin{verbatim}
%		$ ls
%		
%		Applications	Desktop	...
%	\end{verbatim}
% \end{commandline}

\mdfdefinestyle{commandline}{
	leftmargin=10pt,
	rightmargin=10pt,
	innerleftmargin=15pt,
	middlelinecolor=black!50!white,
	middlelinewidth=2pt,
	frametitlerule=false,
	backgroundcolor=black!5!white,
	frametitle={Command Line},
	frametitlefont={\normalfont\sffamily\color{white}\hspace{-1em}},
	frametitlebackgroundcolor=black!50!white,
	nobreak,
}

% Define a custom environment for command-line snapshots
\newenvironment{commandline}{
	\medskip
	\begin{mdframed}[style=commandline]
}{
	\end{mdframed}
	\medskip
}

%----------------------------------------------------------------------------------------
%	FILE CONTENTS ENVIRONMENT
%----------------------------------------------------------------------------------------

% Usage:
% \begin{file}[optional filename, defaults to "File"]
%	File contents, for example, with a listings environment
% \end{file}

\mdfdefinestyle{file}{
	innertopmargin=1.6\baselineskip,
	innerbottommargin=0.8\baselineskip,
	topline=false, bottomline=false,
	leftline=false, rightline=false,
	leftmargin=2cm,
	rightmargin=2cm,
	singleextra={%
		\draw[fill=black!10!white](P)++(0,-1.2em)rectangle(P-|O);
		\node[anchor=north west]
		at(P-|O){\ttfamily\mdfilename};
		%
		\def\l{3em}
		\draw(O-|P)++(-\l,0)--++(\l,\l)--(P)--(P-|O)--(O)--cycle;
		\draw(O-|P)++(-\l,0)--++(0,\l)--++(\l,0);
	},
	nobreak,
}

% Define a custom environment for file contents
\newenvironment{file}[1][File]{ % Set the default filename to "File"
	\medskip
	\newcommand{\mdfilename}{#1}
	\begin{mdframed}[style=file]
}{
	\end{mdframed}
	\medskip
}

%----------------------------------------------------------------------------------------
%	NUMBERED QUESTIONS ENVIRONMENT
%----------------------------------------------------------------------------------------

% Usage:
% \begin{question}[optional title]
%	Question contents
% \end{question}

\mdfdefinestyle{question}{
	innertopmargin=1.2\baselineskip,
	innerbottommargin=0.8\baselineskip,
	roundcorner=5pt,
	nobreak,
	singleextra={%
		\draw(P-|O)node[xshift=1em,anchor=west,fill=white,draw,rounded corners=5pt]{%
		Question \theQuestion\questionTitle};
	},
}

\newcounter{Question} % Stores the current question number that gets iterated with each new question

% Define a custom environment for numbered questions
\newenvironment{question}[1][\unskip]{
	\bigskip
	\stepcounter{Question}
	\newcommand{\questionTitle}{~#1}
	\begin{mdframed}[style=question]
}{
	\end{mdframed}
	\medskip
}

%----------------------------------------------------------------------------------------
%	WARNING TEXT ENVIRONMENT
%----------------------------------------------------------------------------------------

% Usage:
% \begin{warn}[optional title, defaults to "Warning:"]
%	Contents
% \end{warn}

\mdfdefinestyle{warning}{
	topline=false, bottomline=false,
	leftline=false, rightline=false,
	nobreak,
	singleextra={%
		\draw(P-|O)++(-0.5em,0)node(tmp1){};
		\draw(P-|O)++(0.5em,0)node(tmp2){};
		\fill[black,rotate around={45:(P-|O)}](tmp1)rectangle(tmp2);
		\node at(P-|O){\color{white}\scriptsize\bf !};
		\draw[very thick](P-|O)++(0,-1em)--(O);%--(O-|P);
	}
}

% Define a custom environment for warning text
\newenvironment{warn}[1][Warning:]{ % Set the default warning to "Warning:"
	\medskip
	\begin{mdframed}[style=warning]
		\noindent{\textbf{#1}}
}{
	\end{mdframed}
}

%----------------------------------------------------------------------------------------
%	INFORMATION ENVIRONMENT
%----------------------------------------------------------------------------------------

% Usage:
% \begin{info}[optional title, defaults to "Info:"]
% 	contents
% 	\end{info}

\mdfdefinestyle{info}{%
	topline=false, bottomline=false,
	leftline=false, rightline=false,
	nobreak,
	singleextra={%
		\fill[black](P-|O)circle[radius=0.4em];
		\node at(P-|O){\color{white}\scriptsize\bf i};
		\draw[very thick](P-|O)++(0,-0.8em)--(O);%--(O-|P);
	}
}

% Define a custom environment for information
\newenvironment{info}[1][Info:]{ % Set the default title to "Info:"
	\medskip
	\begin{mdframed}[style=info]
		\noindent{\textbf{#1}}
}{
	\end{mdframed}
}

%----------------------------------------------------------------------------------------
%	EQUATION BOXES
%----------------------------------------------------------------------------------------
 % Include the file specifying the document structure and custom commands

%----------------------------------------------------------------------------------------
%	ASSIGNMENT INFORMATION
%----------------------------------------------------------------------------------------

\title{Critical Systems Evaluation} % Title of the assignment

\author{First Homework List\\ \texttt{acn2}} % Author name and email address

\date{CIN, UFPE --- \today} % University, school and/or department name(s) and a date

%----------------------------------------------------------------------------------------

\begin{document}

\maketitle % Print the title

%----------------------------------------------------------------------------------------
%	PROBLEM 8
%----------------------------------------------------------------------------------------
\setcounter{Question}{7}

\begin{question}
	Assume the hazard function of a system is $\lambda (t) = t/250000$. Calculate
	\begin{enumerate}[(a)]
		\item the Reliability of the system at $t = 720 \unit{\hour}$
		\item the MTTF
	\end{enumerate}
\end{question}

\begin{enumerate}[(a)]
    \item The Reliability of a system can be described by the following equation:
    \begin{equation}
        R(t) = e^{-\int_0^t \lambda (t) \odif{t}}
    \end{equation}
    We know that $\lambda (t) = t/(\num{2.5e5})$, thus
    \begin{align*}
        \int_0^t \lambda (t) \odif{t} &= \int_0^t \frac{t}{\num{2.5e5}} \odif{t} \\
        &= \frac{1}{\num{2.5e5}} \int_0^t t \odif{t} \\
        &= \frac{1}{\num{2.5e5}} \cdot \frac{t^2}{2} \\
        &= \frac{t^2}{\num{5e5}}
    \end{align*}
    Therefore,
    \begin{align}
        R(t) = e^{-t^2/({\num{5e5})}} \label{eq:q8_reliability}
    \end{align}
    Assuming $t = 720$,
    \begin{align*}
        \frac{t^2}{\num{5e5}} &=\frac{720^2}{\num{5e5}} = \frac{648}{625}
    \end{align*}
    Thus, we have that
    \begin{empheq}[box=\fbox]{align*}
        R(t=720) &= e^{-648/625} \\
        &\approx \num{0.354}
    \end{empheq}    
    \item The MTTF can be expressed as a function of the Reliability:
    \begin{align}
        MTTF = \int_0^\infty R(t) \odif{t}
    \end{align}
    We'll use the value of $R(t)$ we found on \eqref{eq:q8_reliability}. As this integral is too complex to calculate by hand, we used the Wolfram Alpha tool to compute the result.
    \begin{empheq}[box=\fbox]{align*}
        MTTF &= \int_0^\infty e^{-t^2/({\num{5e5})}} \odif{t} \\
        &\approx 626.657 \text{ hours}
    \end{empheq}
\end{enumerate}

%----------------------------------------------------------------------------------------
%	PROBLEM 9
%----------------------------------------------------------------------------------------

\begin{question}
	If the cumulative rate function of system is $\num{7.5e-7}\times t^3$, compute
	\begin{enumerate}[(a)]
		\item the reliability at $t = 720 \unit{\hour}$
		\item the failure rate at $t = 500 \unit{\hour}$
		\item the MTTF
	\end{enumerate}

    \begin{enumerate}[(a)]
        \item 
    \end{enumerate}

\end{question}

%----------------------------------------------------------------------------------------
%	PROBLEM 10
%----------------------------------------------------------------------------------------

\begin{question}
    Let us assume the failure and the repair rates are equal to $\lambda = \num{1.2e-6}$ and $\mu = \num{1.2e-3}$, respectively.
	\begin{enumerate}[(a)]
		\item Calculate the MTTF and MTBF
		\item Calculate the reliability at $t = 8760 \unit{\hour}$
		\item If the annual downtime is $15\unit{\hour}$, what is the steady-state availability?
		\item What is the effect on the availability if we triple the MTTR?
	\end{enumerate}

\end{question}

\begin{enumerate}[(a)]
    \item As the failure and repair rates are constant, we can define MTTF and MTTR as
    \begin{align*}
        MTTF &= \frac{1}{\lambda} & 
        MTTR &= \frac{1}{\mu}
    \end{align*}
    Furthermore, assuming that $\text{MNRT} \simeq 0$,
    \begin{align*}
        MTBF = MTTF + MTTR = \frac{1}{\lambda} + \frac{1}{\mu} 
    \end{align*}
    Therefore, using the values provided for $\lambda$ and $\mu$,
    \begin{empheq}[box=\fbox]{align*}
        MTTF &= \frac{1}{\num{1.2e-6}} \\
        &\approx \qty{833333.33}{hours} \\
        MTBF &= \frac{1}{\num{1.2e-6}} + \frac{1}{\num{1.2e-3}} \\
        &\approx \qty{834166.67}{hours}
    \end{empheq} 
    \item As the failure rate is constant, we can define the Reliability $R$ as $$R(t) = e^{-\lambda t}$$
    Therefore, using the values provided for $\lambda$ and $t$,
    \begin{empheq}[box=\fbox]{align*}
        R(t) &= e^{\num{1.2e-6} \times 8760} \\
        &\approx \num{0.9895}
    \end{empheq}
    \item The system downtime ($DT$) can be expressed as 
    \begin{equation}
        DT = (1-A) \times T \label{eq:q10_dt}
    \end{equation}
    Where $T$ is the time period, and $A$ is the steady-state availability. We know $DT = \qty{15}{h}$ refers to the annual downtime, so assuming a year of 365 days
    $$T = 365 \times 24 = \qty{8760}{hours}$$
    Plugging the known variables into equation \eqref{eq:q10_dt}
    \begin{align*}
        15 &= (1-A) \times 8760 \\
        \frac{15}{8760} &= 1-A \\
        A &= 1 - \frac{15}{8760} \\
        &\approx \num{0.9982}
    \end{align*}
    Therefore, the steady-state availability $A$ for a system with 15 hours of downtime is
    \begin{empheq}[box=\fbox]{align*}
        A \approx \num{0.9982}
    \end{empheq} 
\end{enumerate}

%----------------------------------------------------------------------------------------

\end{document}
