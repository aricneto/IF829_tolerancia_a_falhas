%%%%%%%%%%%%%%%%%%%%%%%%%%%%%%%%%%%%%%%%%
% Lachaise Assignment
% LaTeX Template
% Version 1.0 (26/6/2018)
%
% This template originates from:
% http://www.LaTeXTemplates.com
%
% Authors:
% Marion Lachaise & François Févotte
% Vel (vel@LaTeXTemplates.com)
%
% License:
% CC BY-NC-SA 3.0 (http://creativecommons.org/licenses/by-nc-sa/3.0/)
% 
%%%%%%%%%%%%%%%%%%%%%%%%%%%%%%%%%%%%%%%%%

%----------------------------------------------------------------------------------------
%	PACKAGES AND OTHER DOCUMENT CONFIGURATIONS
%----------------------------------------------------------------------------------------

\documentclass{article}
\usepackage[brazil]{babel}

%%%%%%%%%%%%%%%%%%%%%%%%%%%%%%%%%%%%%%%%%
% Lachaise Assignment
% Structure Specification File
% Version 1.0 (26/6/2018)
%
% This template originates from:
% http://www.LaTeXTemplates.com
%
% Authors:
% Marion Lachaise & François Févotte
% Vel (vel@LaTeXTemplates.com)
%
% License:
% CC BY-NC-SA 3.0 (http://creativecommons.org/licenses/by-nc-sa/3.0/)
% 
%%%%%%%%%%%%%%%%%%%%%%%%%%%%%%%%%%%%%%%%%

%----------------------------------------------------------------------------------------
%	PACKAGES AND OTHER DOCUMENT CONFIGURATIONS
%----------------------------------------------------------------------------------------

\usepackage{amsmath,amsfonts,stmaryrd,amssymb} % Math packages

\usepackage{enumerate} % Custom item numbers for enumerations

\usepackage[ruled]{algorithm2e} % Algorithms

\usepackage[framemethod=tikz]{mdframed} % Allows defining custom boxed/framed environments

\usepackage{listings} % File listings, with syntax highlighting
\lstset{
	basicstyle=\ttfamily, % Typeset listings in monospace font
}

\usepackage{derivative} % support for derivatives
\usepackage{siunitx} % support for SI units

\usepackage{empheq} 
\usepackage{xcolor}
\definecolor{lightgreen}{HTML}{90EE90}

%----------------------------------------------------------------------------------------
%	DOCUMENT MARGINS
%----------------------------------------------------------------------------------------

\usepackage{geometry} % Required for adjusting page dimensions and margins

\geometry{
	paper=a4paper, % Paper size, change to letterpaper for US letter size
	top=2.5cm, % Top margin
	bottom=3cm, % Bottom margin
	left=2.5cm, % Left margin
	right=2.5cm, % Right margin
	headheight=14pt, % Header height
	footskip=1.5cm, % Space from the bottom margin to the baseline of the footer
	headsep=1.2cm, % Space from the top margin to the baseline of the header
	%showframe, % Uncomment to show how the type block is set on the page
}

%----------------------------------------------------------------------------------------
%	FONTS
%----------------------------------------------------------------------------------------

\usepackage[utf8]{inputenc} % Required for inputting international characters
\usepackage[T1]{fontenc} % Output font encoding for international characters

\usepackage{XCharter} % Use the XCharter fonts

%----------------------------------------------------------------------------------------
%	COMMAND LINE ENVIRONMENT
%----------------------------------------------------------------------------------------

% Usage:
% \begin{commandline}
%	\begin{verbatim}
%		$ ls
%		
%		Applications	Desktop	...
%	\end{verbatim}
% \end{commandline}

\mdfdefinestyle{commandline}{
	leftmargin=10pt,
	rightmargin=10pt,
	innerleftmargin=15pt,
	middlelinecolor=black!50!white,
	middlelinewidth=2pt,
	frametitlerule=false,
	backgroundcolor=black!5!white,
	frametitle={Command Line},
	frametitlefont={\normalfont\sffamily\color{white}\hspace{-1em}},
	frametitlebackgroundcolor=black!50!white,
	nobreak,
}

% Define a custom environment for command-line snapshots
\newenvironment{commandline}{
	\medskip
	\begin{mdframed}[style=commandline]
}{
	\end{mdframed}
	\medskip
}

%----------------------------------------------------------------------------------------
%	FILE CONTENTS ENVIRONMENT
%----------------------------------------------------------------------------------------

% Usage:
% \begin{file}[optional filename, defaults to "File"]
%	File contents, for example, with a listings environment
% \end{file}

\mdfdefinestyle{file}{
	innertopmargin=1.6\baselineskip,
	innerbottommargin=0.8\baselineskip,
	topline=false, bottomline=false,
	leftline=false, rightline=false,
	leftmargin=2cm,
	rightmargin=2cm,
	singleextra={%
		\draw[fill=black!10!white](P)++(0,-1.2em)rectangle(P-|O);
		\node[anchor=north west]
		at(P-|O){\ttfamily\mdfilename};
		%
		\def\l{3em}
		\draw(O-|P)++(-\l,0)--++(\l,\l)--(P)--(P-|O)--(O)--cycle;
		\draw(O-|P)++(-\l,0)--++(0,\l)--++(\l,0);
	},
	nobreak,
}

% Define a custom environment for file contents
\newenvironment{file}[1][File]{ % Set the default filename to "File"
	\medskip
	\newcommand{\mdfilename}{#1}
	\begin{mdframed}[style=file]
}{
	\end{mdframed}
	\medskip
}

%----------------------------------------------------------------------------------------
%	NUMBERED QUESTIONS ENVIRONMENT
%----------------------------------------------------------------------------------------

% Usage:
% \begin{question}[optional title]
%	Question contents
% \end{question}

\mdfdefinestyle{question}{
	innertopmargin=1.2\baselineskip,
	innerbottommargin=0.8\baselineskip,
	roundcorner=5pt,
	nobreak,
	singleextra={%
		\draw(P-|O)node[xshift=1em,anchor=west,fill=white,draw,rounded corners=5pt]{%
		Question \theQuestion\questionTitle};
	},
}

\newcounter{Question} % Stores the current question number that gets iterated with each new question

% Define a custom environment for numbered questions
\newenvironment{question}[1][\unskip]{
	\bigskip
	\stepcounter{Question}
	\newcommand{\questionTitle}{~#1}
	\begin{mdframed}[style=question]
}{
	\end{mdframed}
	\medskip
}

%----------------------------------------------------------------------------------------
%	WARNING TEXT ENVIRONMENT
%----------------------------------------------------------------------------------------

% Usage:
% \begin{warn}[optional title, defaults to "Warning:"]
%	Contents
% \end{warn}

\mdfdefinestyle{warning}{
	topline=false, bottomline=false,
	leftline=false, rightline=false,
	nobreak,
	singleextra={%
		\draw(P-|O)++(-0.5em,0)node(tmp1){};
		\draw(P-|O)++(0.5em,0)node(tmp2){};
		\fill[black,rotate around={45:(P-|O)}](tmp1)rectangle(tmp2);
		\node at(P-|O){\color{white}\scriptsize\bf !};
		\draw[very thick](P-|O)++(0,-1em)--(O);%--(O-|P);
	}
}

% Define a custom environment for warning text
\newenvironment{warn}[1][Warning:]{ % Set the default warning to "Warning:"
	\medskip
	\begin{mdframed}[style=warning]
		\noindent{\textbf{#1}}
}{
	\end{mdframed}
}

%----------------------------------------------------------------------------------------
%	INFORMATION ENVIRONMENT
%----------------------------------------------------------------------------------------

% Usage:
% \begin{info}[optional title, defaults to "Info:"]
% 	contents
% 	\end{info}

\mdfdefinestyle{info}{%
	topline=false, bottomline=false,
	leftline=false, rightline=false,
	nobreak,
	singleextra={%
		\fill[black](P-|O)circle[radius=0.4em];
		\node at(P-|O){\color{white}\scriptsize\bf i};
		\draw[very thick](P-|O)++(0,-0.8em)--(O);%--(O-|P);
	}
}

% Define a custom environment for information
\newenvironment{info}[1][Info:]{ % Set the default title to "Info:"
	\medskip
	\begin{mdframed}[style=info]
		\noindent{\textbf{#1}}
}{
	\end{mdframed}
}

%----------------------------------------------------------------------------------------
%	EQUATION BOXES
%----------------------------------------------------------------------------------------
 % Include the file specifying the document structure and custom commands

%----------------------------------------------------------------------------------------
%	ASSIGNMENT INFORMATION
%----------------------------------------------------------------------------------------

\title{Critical Systems Evaluation} % Title of the assignment

\author{First Homework List\\ Equipe: \texttt{aasb2, acn2, pvoa, vbmp}} % Author name and email address

\date{CIn, UFPE --- \today} % University, school and/or department name(s) and a date

%----------------------------------------------------------------------------------------

\begin{document}

\maketitle % Print the title

\begin{question}
    Explain the concepts of 
    \begin{enumerate}[{(a)}]
        \item fault, error, and failure
        \item reliability, steady-state availability and instantaneous availability
        \item  hazard and cumulative hazard functions
    \end{enumerate}
\end{question}

\begin{enumerate}[{(a)}]
    \item \textbf{fault} is the adjudged or hypothesized cause of an error, \textbf{error} is a state of a component of a system (a system substate) that may cause a subsequent failure and a \textbf{failure} is what occurs when an error reaches the system interface and alters the service.
    \item \textbf{reliability} is the probability that the system S does not fail up to time t, \textbf{steady-state availability} is something that is possible to quantify when system approaches stationary states, the formula for it is $A=\lim_{t \to \infty} A(t), t\ge0$ and \textbf{instantaneous availability} is the probability that the system is operational at time t.
    \item \textbf{hazard function} is the probability of the system S failing during the interval $[t,t+\delta t]$ if it has survived to the time t and \textbf{cumulative hazard function} is the total hazard until the given point in time.
\end{enumerate}
    
\begin{question}
Consider a database system with time to failure distribution represented
by the $CDF$ $F(t)=1+e^{-0.0005t}-2e^{-0.00025t}$.  
    \begin{enumerate}[{(a)}]
        \item Calculate the reliability at $t=5,500h$ 
        \item and compute the $MTTF$,
        \item obtain the system hazard function $\lambda(t)$  After, calculate $\lambda(t)$ at $t=5,500h$
    \end{enumerate}
\end{question}

\begin{enumerate}[{(a)}]
    \item The reliability of a system can be described by the following equation:
    \begin{equation}
        R(t) = 1 - F(t)
    \end{equation}
    We know that $F(t)=1+e^{-0.0005t}-2e^{-0.00025t}$, thus
    \begin{align*}
         R(t)&=1-(1+e^{-0.0005t}-2e^{-0.00025t}) \\
         &=1-1-e^{-0.0005t}+2e^{-0.00025t}  \\
        &=-e^{-0.0005t}+2e^{-0.00025t} \\
        &=2e^{-0.00025t}-e^{-0.0005t}
    \end{align*}
    For $t=5,500h$  we have
   \begin{empheq}[box=\fbox]{align*}
        R(t=5500)=2e^{-0.00025\cdot5500}-e^{-0.0005\cdot5500} \approx 0,442
    \end{empheq}
    
    \item The Mean Time to Failure ($MTTF$) can be described by the following equation
    \begin{equation}
        \int_{0}^{\infty} R(t) dt
    \end{equation}
    We know that $R(t)=2e^{-0.00025t}-e^{-0.0005t}$, thus
    \begin{empheq}[box=\fbox]{align*}
        \int_{0}^{\infty} R(t) dt &= \int_{0}^{\infty}2e^{-0.00025t}-e^{-0.0005t}dt \\
        &= \int_{0}^{\infty}2e^{-0.00025t}dt - \int_{0}^{\infty}e^{-0.0005t}dt \\
        &= 8000 - 4000 \\
        &= 4000h
    \end{empheq}

    \item We know that $\lambda(t)$ can be described as
    \begin{equation}
        \lambda(t) = -\frac{dR(t)}{dt} \cdot \frac{1}{R(t)}
    \end{equation}
    and we know that $R(t)=2e^{-0.00025t}-e^{-0.0005t}$, thus
    \begin{align*}
        \lambda(t) &= -\frac{dR(t)}{dt} \cdot \frac{1}{R(t)} \\
        &=- \frac{\frac{d}{dt}(2e^{-0.00025t}-e^{-0.0005t})}{{2e^{-0.00025t}-e^{-0.0005t}}}\\
    \end{align*}
    so
    \begin{empheq}[box=\fbox]{align*}
        \lambda(t)=-\frac{0.0005e^{-0.0005t}-0.0005e^{-0.00025t}}{2e^{-0.00025t}-e^{-0.0005t}}
    \end{empheq}
    Substituting $t=5500$ on the obtained system hazard funciton we have
    \begin{empheq}[box=\fbox]{align*}
        \lambda(t=5500) &= -\frac{0.0005e^{-0.0005\cdot5500}-0.0005e^{-0.00025\cdot5500}}{2e^{-0.00025\cdot5500}-e^{-0.0005\cdot5500}} \\
        &= 0.0002138 \\
        &= 2.138 \cdot 10^{-4}/h
    \end{empheq}
\end{enumerate}

\begin{question}
Assume the system reliability in $t = 720h$ is $7,500 DPM$. Calculate the probability of failure by $t=720 h$
\end{question}
The reliability of a system can be described by the following equation
    \begin{equation}
        R(t) = 1 - DPM\cdot10^{-6}
    \end{equation}
And the probability of failure of a system can be described as
    \begin{equation}
        F(t) = 1 - R(t)
    \end{equation}
And the equation to convert $R(t)$ to $DPM$ is
    \begin{equation}
        R(t) = 1 - DPM\cdot10^{-6}
    \end{equation}
For $t=720h$ and DPM = $7500DPM$ we have
    \begin{empheq}[box=\fbox]{align*}
        F(t=720)&=1-(1-R(t=720)) \\
        &= 1 - 1+(7500\cdot10^{-6})\\
        &= 7.5\cdot10^{-3}
    \end{empheq}

\begin{question}
    Assume the reliability of a system that is represented by
    \[
    R(t) =
    \begin{cases}
        \frac{1-0.0004t}{0.96} & \text{if } 100 \leq t \leq 2500 \\
        0 & \text{if } t > 2500
    \end{cases}
    \]
    Obtain the density function of the time to failure and calculate the $MTTF$ and $MedTTF$.
\end{question}
The probability of failure of a system cam be described by the following equation
    \begin{equation}
        F(t)=1-R(t)
    \end{equation}
Therefore,
    \[
    F(t) =
    \begin{cases}
        1 - \frac{1-0.0004t}{0.96} & \text{if } 100 \leq t \leq 2500 \\
        1 & \text{if } t > 2500
    \end{cases}
    \]
The density function of the time to failure can be described by the following equation
    \begin{equation}
        f(t)=\frac{d}{dt}F(t)
    \end{equation}
Thus,
    \[
    f(t) =
    \begin{cases}
        0.0004 & \text{if } 100 \leq t \leq 2500 \\
        0 & \text{if } t > 2500
    \end{cases}
    \]
The mean time to fail ($MTTF$) can be described as
    \begin{equation}
        MTTF=\int_{0}^{\infty}tf(t)dt
    \end{equation}   
Therefore,
    \begin{empheq}[box=\fbox]{align*}
        MTTF&=\int_{100}^{2500}0.004t dt+ \int_{2500}^{\infty} 0\cdot t dt\\
        &=12500+0 \\
        &=12500h
    \end{empheq}
The median time to failure ( $MedTTF$ ) is  defined by the time instant, t, when $R(t)=0.5$, therefore
    \begin{align*}
        &R(t)=0.5=\frac{1=0.0004t}{0.96} \\
        & 1-0.004t = 0.96\cdot0.50 = 0.48 \\
        & -0.004t = 0.48 - 1 \\
        & -0.004t = -0.052 \\
        & t = \frac{0.052}{0.004} \\
        & t = 130h 
    \end{align*}
So the value for $MedTTF$ is
    \begin{empheq}[box=\fbox]{align*}
        MedTTF = 130h
    \end{empheq}

%----------------------------------------------------------------------------------------
%	PROBLEM 5
%----------------------------------------------------------------------------------------
\begin{question}
	The time to failure of web server is exponentially distributed with rate \( 5 \times 10^{-4} \) \textit{fph}. 
    \begin{enumerate}[(a)]
    \item What is the probability that the system does not fail in the first two months? 
    \item Assume the time to repair rate is constant and equal to \( 5 \times 10^{-2} \). Estimate the probability of being operational at \( 720h \).
    \item Compute the steady-state availability.
    \end{enumerate}
\end{question}

\begin{enumerate}[(a)]
    \item The exponential distribution has a constant Hazard function. Over a span of two months, equivalent to 1440 hours, we can explore this characteristic using the reliability function.
\begin{gather*}
    R(t) = e^{-\lambda t};\quad \lambda = 5*10^{-4};\quad   t = 1440 \\
    R(1440) = e^{-5 \times 10^4 \times 1440} = e^{-0,72} \\
    \fbox{$R(1440) = 0.4867 = 48,67\%$}
\end{gather*}
    \item We can calculate the instantaneous availability using the equation: 
    \begin{equation*}
        A(t) = \frac{\mu}{\mu + \lambda} + \frac{\lambda}{\mu + \lambda}  e^{-(\lambda + \mu) t} 
    \end{equation*}
    Thus, with
    \begin{gather*}
        \lambda = 5 \times 10^{-4} \qquad \mu = 5 \times 10^{-2}
    \end{gather*}
    We have
    \begin{empheq}[box=\fbox]{align*}
        A(720) &= \frac{5 \times 10^{-2}}{5 \times 10^{-4} + 5 \times 10^{-2}} + \frac{5 \times 10^{-4}}{5 \times 10^{-4} + 5 \times 10^{-2}} e^{-(5 \times 10^{-4} + 5 \times 10^{-2}) 720} \\
        &\approx 99\%
    \end{empheq}
    \item In scenarios where the repair rate is constant, the hazard function becomes a valuable companion in calculating availability. This dynamic relationship,
 \begin{equation}
     A(t) = \frac{\mu}{\mu + \lambda} \notag
 \end{equation}
Underscores the significance of considering both reliability and the constant repair rate to gain a comprehensive understanding of system performance over time.
\begin{gather*}
    \lambda = 5 \times 10^{-4} \\ 
    \mu = 5 \times 10^{-2} \\
    A = \frac{5 \times 10^{-2}}{5 \times 10^{-2} + 5 \times 10^{-4}} \\
    \fbox{$A = 99\%$}
\end{gather*}

\end{enumerate}
\newpage

%----------------------------------------------------------------------------------------
%	PROBLEM 6
%----------------------------------------------------------------------------------------
\begin{question}
	The time to failure of the system is distributed according to a Weibull distribution with shape parameter equal to \( \alpha = 2 \), scale parameter equal to \( \beta = 1000h \), and location parameter equal to zero.
\begin{enumerate}[(a)]
    \item What is the reliability at \( 720h \)?
    \item Calculate the \( MTTF \).
	\end{enumerate}
\end{question}

\begin{enumerate}[(a)]
    \item Reliability function for Weibull distribution
\begin{gather*}
    R(t) = e^{-\left(\frac{t-\mu}{\beta}\right)^\alpha} \\
    t = 720h; \quad \beta = 1000; \quad \alpha = 2; \quad \mu = 0
\end{gather*}
Therefore,
\begin{align*}
    R(720) &= e^{\left(\frac{720}{1000}\right)^2} \\
    &= e^{-0.722} \\
    &= 0.5954 \\
    &\fbox{$R(720) = 59,54\%$}
\end{align*}

    \item The Mean Time To Failure (MTTF) is calculated using the formula: 
\begin{gather*}
    MTTF = \beta \times \Gamma(1 + \frac{1}{\alpha}) + \mu \\
    MTTF = 10^3 \times \Gamma( 1 + \frac{1}{2}) + 0 \\
    \fbox{$MTTF = 10^3 \times 0,886 = 886 h$}
\end{gather*}
\end{enumerate}

%----------------------------------------------------------------------------------------
%	PROBLEM 7
%----------------------------------------------------------------------------------------
\begin{question}
	Now, consider the distribution of the previous example has a shape parameter equal to \( \alpha = 0.8 \).
    \begin{enumerate}[(a)] 
    \item What is the reliability at \( 720h \)?
    \item Calculate the \( MTTF \).
	\end{enumerate}
\end{question}

\begin{enumerate}[(a)]
    \item
\begin{gather*}
    R(t) = e^{-\left(\frac{t-\mu}{\beta}\right)^\alpha} \\
    t = 720h;\quad \beta = 1000;\quad \alpha = 0,8;\quad \mu = 0 \\
    R(720) = e^{(0,72)^{0,8}} = 0,4639 \\
    \fbox{$R(720) = 46,39\%$}
\end{gather*}
    \item
\begin{gather*}
    MTTF = \beta \times \Gamma(1 + \frac{1}{\alpha}) + \mu \\
    MTTF = 10^3 \times \Gamma(1 + \frac{1}{0,8}) + 0 \\
    \fbox{$MTTF = 10^3 \times 1,133 = 1133 h$}
\end{gather*}
\end{enumerate}

%----------------------------------------------------------------------------------------
%	PROBLEM 8
%----------------------------------------------------------------------------------------
\setcounter{Question}{7}

\begin{question}
	Assume the hazard function of a system is $\lambda (t) = t/250000$. Calculate
	\begin{enumerate}[(a)]
		\item the reliability of the system at $t = 720 \unit{\hour}$
		\item the $MTTF$
	\end{enumerate}
\end{question}

\begin{enumerate}[(a)]
    \item The reliability of a system can be described by the following equation:
    \begin{equation}
        R(t) = e^{-\int_0^t \lambda (t) \odif{t}}
    \end{equation}
    We know that $\lambda (t) = t/(\num{2.5e5})$, thus
    \begin{align*}
        \int_0^t \lambda (t) \odif{t} &= \int_0^t \frac{t}{\num{2.5e5}} \odif{t} \\
        &= \frac{1}{\num{2.5e5}} \int_0^t t \odif{t} \\
        &= \frac{1}{\num{2.5e5}} \cdot \frac{t^2}{2} \\
        &= \frac{t^2}{\num{5e5}}
    \end{align*}
    Therefore,
    \begin{align}
        R(t) = e^{-t^2/({\num{5e5})}} \label{eq:q8_reliability}
    \end{align}
    Assuming $t = 720$,
    \begin{align*}
        \frac{t^2}{\num{5e5}} &=\frac{720^2}{\num{5e5}} = \frac{648}{625}
    \end{align*}
    Thus, we have that
    \begin{empheq}[box=\fbox]{align*}
        R(t=720) &= e^{-648/625} \\
        &\approx \num{0.354}
    \end{empheq}    
    \item The MTTF can be expressed as a function of the reliability:
    \begin{align}
        MTTF = \int_0^\infty R(t) \odif{t} \label{eq:MTTF}
    \end{align}
    We'll use the value of $R(t)$ we found on \eqref{eq:q8_reliability}. As this integral is too complex to calculate by hand, we used the Wolfram Alpha tool to compute the result.
    \begin{empheq}[box=\fbox]{align*}
        MTTF &= \int_0^\infty e^{-t^2/({\num{5e5})}} \odif{t} \\
        &\approx 626.657 \text{ hours}
    \end{empheq}
\end{enumerate}
\newpage

%----------------------------------------------------------------------------------------
%	PROBLEM 9
%----------------------------------------------------------------------------------------

\begin{question}
	If the cumulative rate function of system is $\num{7.5e-7}\times t^3$, compute
	\begin{enumerate}[(a)]
		\item the reliability at $t = 720 \unit{\hour}$
		\item the failure rate at $t = 500 \unit{\hour}$
		\item the MTTF
	\end{enumerate}
\end{question}

\begin{enumerate}[(a)]
    \item The reliability can be expressed as a function of the cumulative hazard rate $H(t)$
    \begin{align}
        H(t) &= \int_0^t \lambda(t) \odif{t} \notag\\
        R(t) &= e^{-\int_0^t \lambda(t) \odif{t}} \notag\\
        \therefore \notag\\
        R(t) &= e^{-H(t)} \label{eq:q9_rtht}
    \end{align}
    At $t=\qty{720}{h}$, we have
    \begin{align*}
        H(t) &= \num{7.5e-7} \times 720^3 \\
        &\approx 279.936
    \end{align*}
    We can then find the reliability at $t=\qty{720}{h}$ using equation \eqref{eq:q9_rtht}
    \begin{empheq}[box=\fbox]{align*}
        R(t) &= e^{-279.936} \\
        &\approx \num{2.662e-122}
    \end{empheq}

    \item By deriving the cumulative hazard rate, we can find the formula for the failure rate
    \begin{align*}
        h(t) &= \odv{}{t} (\num{7.5e-7} \times t^3) \\
        &= \num{2.25e-6} \times t^2
    \end{align*}
    At $t=\qty{500}{h}$ we have
    \begin{empheq}[box=\fbox]{align*}
        h(t) &= \num{2.25e-6} \times 500^2 \\
        &= \num{0.5625}
    \end{empheq}

    \item We'll use formulas \eqref{eq:MTTF} and \eqref{eq:q9_rtht} to compute the MTTF. Once again, we use the Wolfram Alpha tool to compute the result, as this integral is too complex to evaluate by hand
    \begin{empheq}[box=\fbox]{align*}
        MTTF &= \int_0^\infty e^{-\num{7.5e-7} \times t^3} \odif{t} \\
        &\approx \num{98.2851}
    \end{empheq}
\end{enumerate}

\pagebreak
%----------------------------------------------------------------------------------------
%	PROBLEM 10
%----------------------------------------------------------------------------------------

\begin{question}
    Let us assume the failure and the repair rates are equal to $\lambda = \num{1.2e-6}$ and $\mu = \num{1.2e-3}$, respectively.
	\begin{enumerate}[(a)]
		\item Calculate the MTTF and MTBF
		\item Calculate the reliability at $t = 8760 \unit{\hour}$
		\item If the annual downtime is $15\unit{\hour}$, what is the steady-state availability?
		\item What is the effect on the availability if we triple the MTTR?
	\end{enumerate}

\end{question}

\begin{enumerate}[(a)]
    \item As the failure and repair rates are constant, we can define MTTF and MTTR as
    \begin{align}
        MTTF &= \frac{1}{\lambda} & 
        MTTR &= \frac{1}{\mu} \label{eq:q10_mttf}
    \end{align}
    Furthermore, assuming that $\text{MNRT} \simeq 0$,
    \begin{align*}
        MTBF = MTTF + MTTR = \frac{1}{\lambda} + \frac{1}{\mu} 
    \end{align*}
    Therefore, using the values provided for $\lambda$ and $\mu$,
    \begin{empheq}[box=\fbox]{align*}
        MTTF &= \frac{1}{\num{1.2e-6}} \\
        &\approx \qty{833333.33}{hours} \\
        MTBF &= \frac{1}{\num{1.2e-6}} + \frac{1}{\num{1.2e-3}} \\
        &\approx \qty{834166.67}{hours}
    \end{empheq} 
    \item As the failure rate is constant, we can define the reliability $R$ as $$R(t) = e^{-\lambda t}$$
    Therefore, using the values provided for $\lambda$ and $t$,
    \begin{empheq}[box=\fbox]{align*}
        R(t) &= e^{\num{1.2e-6} \times 8760} \\
        &\approx \num{0.9895}
    \end{empheq}
    \item The system downtime ($DT$) can be expressed as 
    \begin{equation}
        DT = (1-A) \times T \label{eq:q10_dt}
    \end{equation}
    Where $T$ is the time period, and $A$ is the steady-state availability. We know $DT = \qty{15}{h}$ refers to the annual downtime, so assuming a year of 365 days
    $$T = 365 \times 24 = \qty{8760}{hours}$$
    Plugging the known variables into equation \eqref{eq:q10_dt}
    \begin{align*}
        15 &= (1-A) \times 8760 \\
        \frac{15}{8760} &= 1-A \\
        A &= 1 - \frac{15}{8760} \\
        &\approx \num{0.9982}
    \end{align*}
    Therefore, the steady-state availability $A$ for a system with 15 hours of downtime is
    \begin{empheq}[box=\fbox]{align*}
        A \approx \num{0.9982}
    \end{empheq} 
    \item We first calculate the original system's availability $A$, given by the formula
    \begin{equation}
        A = \frac{MTTF}{MTTF + MTTR} = \frac{\mu}{\lambda + \mu} \label{eq:q10avail}
    \end{equation}
    From the equations in \eqref{eq:q10_mttf}, and the $MTTF$ we found on item (a), we know that
    \begin{align*}
        MTTF &= \frac{1}{\num{1.2e-6}} & MTTR &= \frac{1}{\num{1.2e-3}}
    \end{align*}
    The original system's availability $A_1$ can then be calculated using \eqref{eq:q10avail}
    \begin{align*}
        A_1 &= \frac{\frac{1}{\num{1.2e-6}}}{\frac{1}{\num{1.2e-6}} + \frac{1}{\num{1.2e-3}}} \\
        &\approx 0.9990
    \end{align*}
    We now triple the MTTR and calculate the new system's availability $A_2$
    \begin{align*}
        A_2 &= \frac{\frac{1}{\num{1.2e-6}}}{\frac{1}{\num{1.2e-6}} + \frac{3}{\num{1.2e-3}}} \\
        &\approx 0.9970
    \end{align*}
    We can then calculate the impact of the new MTTR on the original availability $A_1$ in relative terms
    \begin{empheq}[box=\fbox]{align*}
    1 - \frac{A_2}{A_1} = \num{0.002} = 0.2\%
    \end{empheq}
    Therefore, tripling the MTTR brought down the availability $A_1$ of the system by 0.2\%

\end{enumerate}

%----------------------------------------------------------------------------------------
%	PROBLEM 11
%----------------------------------------------------------------------------------------

\begin{question}
    Consider the distribution of the time to failure of a database system is represented by a logistic distribution with parameters \(\mu = 600h\) and \(\beta = 25h\). 

\begin{enumerate}[(a)]
    \item What is the probability of survival at \(t = 500h\)?
    \item Calculate \( \text{Med}TTF \).
\end{enumerate}
\large\[ F(t) = \frac{1}{e^{-\frac{t-\mu}{\beta}} + 1} \]
\end{question}

\begin{enumerate}[(a)]
    \item Utilizando a expressão dada na questão para a função de distribuição cumulativa complementar temos que:
    \[F(t) = \frac{1}{1+e^{-\frac{t-\mu}{\beta}}}\]
    Como temos que $t = 500h$, $\mu = 600h$ e $\beta = 25h$
    \begin{empheq}[box=\fbox]{align*}
        F(500) &= \frac{1}{1+e^{-\frac{500-600}{25}}} \\
        &= \frac{1}{1+e^4} \\
        &= 0.01798 \\
        &= 17.98\%
    \end{empheq}
    Logo, temos que a probabilidade de sobrevivência desse sistema nesse determinado tempo é de \fbox{$17.98\%$}
    \item Para calcularmos o $MedTTF$ devemos igualar função de distribuição cumulativa complementar ao valor de $0.5$ ($50\%$)
    \begin{gather*}
        \frac{1}{1+e^{-\frac{t-\mu}{\beta}}} = 0.5 \\
        1 = 0.5\times(1+e^{-\frac{t-\mu}{\beta}}) \\
        1 = 0.5 + 0.5e^{-\frac{t-\mu}{\beta}} \\
        0.5 = 0.5e^{-\frac{t-\mu}{\beta}}
    \end{gather*}
    Utilizando de manipulação algébrica e dividindo ambos os lados por $0.5$ para simplificar a expressão
    \[1 = e^{-\frac{t-\mu}{\beta}}\]
    Usando do logaritmo natural em ambos os lados chegamos em
    \[ln(1) = -\frac{t-\mu}{\beta}\]
    Como o ln(1) = 0 ficamos com
    \[0 = -\frac{t-\mu}{\beta}\]
    Por fim, chegamos em
    \[t=\mu\]
    E podemos concluir que o MedTTF é
    \begin{empheq}[box=\fbox]{align*}
        MedTTF = t = 600h
    \end{empheq}

\end{enumerate}

%----------------------------------------------------------------------------------------
%	PROBLEM 12
%----------------------------------------------------------------------------------------

\begin{question}
    The time to failure of system is distributed according a Dagum distribution with shape parameters \( p = 0.5 \), \( a = 2 \), and the scale parameter equal to \( b = 2000h \). 
    \begin{enumerate}[(a)]
        \item What is the reliability at \( 720h \)?
        \item Calculate the \( MTTF \). 
    \end{enumerate}
    Change \( a = 4 \) and calculate 
    \begin{enumerate}[(c)]
            \item the reliability at \( 720h \), and 
            \item Calculate the \( MTTF \).
    \end{enumerate}
    
    \[ \lambda(t) = \frac{ap}{t \left( \left( \frac{t}{b} \right)^a + 1 \right) \left( \left( \left( \frac{b}{t} \right)^a + 1 \right)^p - 1 \right)}, \quad t \geq 0. \]
\end{question}

\begin{enumerate}[(a)]
    \item A reliability function será dada pelo seguinte complemento:
    \[R(t) = 1- F(t)\]
    Dado que
    \[F(t) = \int_0^t \frac{ap}{t \left( \left( \frac{t}{b} \right)^a + 1 \right) \left( \left( \left( \frac{b}{t} \right)^a + 1 \right)^p - 1 \right)}\]
    Substituindo os parâmetros informados na questão ($p = 0.5$; $a = 2$; $b = 2000h$ e $t = 720h$)
    \begin{equation*}
        F(720) = \int_0^{720} \frac{1}{t \left( \left( \frac{t}{2000} \right)^2 + 1 \right) \left( \left( \left( \frac{2000}{t} \right)^2 + 1 \right)^{0.5} - 1 \right)} 
    \end{equation*}
    Resolvendo a integral
    \begin{align}
        F(720) &= ln(2000) - ln(80\sqrt{706} - 720) \notag\\ 
        &\approx 0.3526453302526999 \label{eq:q12f1}
    \end{align}
    Substituindo o valor encontrado para a reliability function
    \begin{empheq}[box=\fbox]{align*}
        R(720) &= 1 - F(720) \\
        &= 1 - 0.3526453302526999 \\
        &= 0.6473546697473001 \\
        &\approx 64.73\%
    \end{empheq}
    \item Utilizaremos da relação entre a reliability function e o MTTF dada por
    \[MTTF = \int_{0}^{t}R(t) \odif{t}\]
    Como $R(t)$ é dado por
    \[R(t) = 1 - F(t)\]
    Temos que
    \[MTTF = \int_{0}^{t} 1 - F(t)\]
    Como calculado anteriormente em \eqref{eq:q12f1},
    \[F(720) \approx 0.3526453302526999\]
    Logo,
    \begin{empheq}[box=\fbox]{align*}
        MTTF &= \int_{0}^{720} 1 - 0.3526453302526999 \odif{t} \\
        &= \int_{0}^{720} 0.6473546697473001 \odif{t} \\
        &= \num{466.1}h \\
    \end{empheq}
    \item Refazendo os cálculos com o parâmetro $a$ alterado, temos que ($p = 0.5$; $a = 4$; $b = 2000h$ e $t = 720h$)
    \[F(t) = \int_0^{720} \frac{2}{t \left( \left( \frac{t}{2000} \right)^4 + 1 \right) \left( \left( \left( \frac{2000}{t} \right)^4 + 1 \right)^{0.5} - 1 \right)}\]
    Resolvendo a integral
    \begin{align}
        F(720) &= \ln{(4000000)} - \ln{\left(6400 \sqrt{397186} - 518400\right)} \notag\\
        &\approx 0.1292399179574557 \label{eq:q12df}
    \end{align}
    Temos então
    \begin{empheq}[box=\fbox]{align*}
        R(720) &= 1 - F(720) \\
        &= 1 - 0.1292399179574557 \\
        &= 0.8707600820425443 \\
        &\approx 87.07\%
    \end{empheq}
    \item De semelhante modo que fizemos para o item (b)
    \[MTTF = \int_{0}^{t} 1 - F(t) \odif{t}\]
    Como definido anteriormente em \eqref{eq:q12df}, para $a = 4$ temos
    \[F(720) \approx 0.1292399179574557\]
    Logo,
    \begin{empheq}[box=\fbox]{align*}
        MTTF &= \int_{0}^{720} 1 - 0.1292399179574557 \odif{t} \\
        &= \int_{0}^{720} 0.8707600820425443 \odif{t} \\
        &= \num{626.947}h
    \end{empheq}
\end{enumerate}

%----------------------------------------------------------------------------------------
%	PROBLEM 12
%----------------------------------------------------------------------------------------

\begin{question}
    Assuming the Dagum distribution depicted in the previous exercise, plot the hazard function considering the parameter \(a\) with respective values: 1, 2, and 4. Discuss the results.
\end{question}

\begin{figure}[H]
    \centering
    \begin{tikzpicture}
    \begin{axis}[
        title={Graph of $\lambda(t)$, $a = 1$},
        xlabel={$t$},
        ylabel={$\lambda(t)$},
        grid=major,
        xmin=0, xmax=100,
        ymin=0, ymax=0.05, % Adjust the y-axis range as needed
        domain=0.001:100, % Avoid division by zero at t=0
        samples=100,
        unbounded coords=jump % This is to handle any undefined points in the graph
    ]
    
    % Lambda function with a=1, b=2000, p=0.5
    \addplot[blue, line width=1.5pt] {1 * 0.5 / (x * ((x/2000)^1 + 1) * (((2000/x)^1 + 1)^0.5 - 1))};
    \end{axis}
    \end{tikzpicture}
\end{figure}

Podemos observar que quando a = 1 temos um gráfico do tipo monotone decreasing hazard function. Isso nos transmite que um “produto” desse tipo usado é melhor e mais confiável do que um novo.

\begin{figure}[H]
    \centering
    \begin{tikzpicture}
    \begin{axis}[
        title={Graph of $\lambda(t)$, $a = 2$},
        xlabel={$t$},
        ylabel={$\lambda(t)$},
        grid=major,
        xmin=0, xmax=100,
        ymin=0, ymax=0.05, % Adjust the y-axis range as needed
        domain=0.001:100, % Avoid division by zero at t=0
        samples=100,
        unbounded coords=jump % This is to handle any undefined points in the graph
    ]
    
    % Lambda function with a=1, b=2000, p=0.5
    \addplot[blue, line width=1.5pt] {2 * 0.5 / (x * ((x/2000)^2 + 1) * (((2000/x)^2 + 1)^0.5 - 1))};
    \end{axis}
    \end{tikzpicture}
\end{figure}

\begin{figure}[H]
    \centering
    \begin{tikzpicture}
    \begin{axis}[
        title={Graph of $\lambda(t)$, $a = 4$},
        xlabel={$t$},
        ylabel={$\lambda(t)$},
        grid=major,
        xmin=0, xmax=100,
        ymin=0, ymax=0.05, % Adjust the y-axis range as needed
        domain=0.001:100, % Avoid division by zero at t=0
        samples=100,
        unbounded coords=jump % This is to handle any undefined points in the graph
    ]
    
    % Lambda function with a=1, b=2000, p=0.5
    \addplot[blue, line width=1.5pt] {(4 * 0.5) / (x * ((x/2000)^4 + 1) * (((2000/x)^4 + 1)^(0.5) - 1))};
    \end{axis}
    \end{tikzpicture}
\end{figure}

Para $a = 2$ e $a = 4$ temos a visualização do tipo constante. Uma vez que sua probabilidade de falha não vai ser alterada de acordo com o tempo, em toda sua vida útil o tempo de falha se mantém constante.
\end{document}
