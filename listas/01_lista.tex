%%%%%%%%%%%%%%%%%%%%%%%%%%%%%%%%%%%%%%%%%
% Lachaise Assignment
% LaTeX Template
% Version 1.0 (26/6/2018)
%
% This template originates from:
% http://www.LaTeXTemplates.com
%
% Authors:
% Marion Lachaise & François Févotte
% Vel (vel@LaTeXTemplates.com)
%
% License:
% CC BY-NC-SA 3.0 (http://creativecommons.org/licenses/by-nc-sa/3.0/)
% 
%%%%%%%%%%%%%%%%%%%%%%%%%%%%%%%%%%%%%%%%%

%----------------------------------------------------------------------------------------
%	PACKAGES AND OTHER DOCUMENT CONFIGURATIONS
%----------------------------------------------------------------------------------------

\documentclass{article}

\input{structure.tex} % Include the file specifying the document structure and custom commands

%----------------------------------------------------------------------------------------
%	ASSIGNMENT INFORMATION
%----------------------------------------------------------------------------------------

\title{Critical Systems Evaluation} % Title of the assignment

\author{First Homework List\\ \texttt{acn2}} % Author name and email address

\date{CIN, UFPE --- \today} % University, school and/or department name(s) and a date

%----------------------------------------------------------------------------------------

\begin{document}

\maketitle % Print the title

%----------------------------------------------------------------------------------------
%	PROBLEM 8
%----------------------------------------------------------------------------------------
\setcounter{Question}{7}

\begin{question}
	Assume the hazard function of a system is $\lambda (t) = t/250000$. Calculate
	\begin{enumerate}[(a)]
		\item the Reliability of the system at $t = 720 \unit{\hour}$
		\item the MTTF
	\end{enumerate}
\end{question}

\begin{enumerate}[(a)]
    \item The Reliability of a system can be described by the following equation:
    \begin{equation}
        R(t) = e^{-\int_0^t \lambda (t) \odif{t}}
    \end{equation}
    We know that $\lambda (t) = t/(\num{2.5e5})$, thus
    \begin{align*}
        \int_0^t \lambda (t) \odif{t} &= \int_0^t \frac{t}{\num{2.5e5}} \odif{t} \\
        &= \frac{1}{\num{2.5e5}} \int_0^t t \odif{t} \\
        &= \frac{1}{\num{2.5e5}} \cdot \frac{t^2}{2} \\
        &= \frac{t^2}{\num{5e5}}
    \end{align*}
    Therefore,
    \begin{align}
        R(t) = e^{-t^2/({\num{5e5})}} \label{eq:q8_reliability}
    \end{align}
    Assuming $t = 720$,
    \begin{align*}
        \frac{t^2}{\num{5e5}} &=\frac{720^2}{\num{5e5}} = \frac{648}{625}
    \end{align*}
    Thus, we have that
    \begin{empheq}[box=\fbox]{align*}
        R(t=720) &= e^{-648/625} \\
        &\approx \num{0.354}
    \end{empheq}    
    \item The MTTF can be expressed as a function of the Reliability:
    \begin{align}
        MTTF = \int_0^\infty R(t) \odif{t}
    \end{align}
    We'll use the value of $R(t)$ we found on \eqref{eq:q8_reliability}. As this integral is too complex to calculate by hand, we used the Wolfram Alpha tool to compute the result.
    \begin{empheq}[box=\fbox]{align*}
        MTTF &= \int_0^\infty e^{-t^2/({\num{5e5})}} \odif{t} \\
        &\approx 626.657 \text{ hours}
    \end{empheq}
\end{enumerate}

%----------------------------------------------------------------------------------------
%	PROBLEM 9
%----------------------------------------------------------------------------------------

\begin{question}
	If the cumulative rate function of system is $\num{7.5e-7}\times t^3$, compute
	\begin{enumerate}[(a)]
		\item the reliability at $t = 720 \unit{\hour}$
		\item the failure rate at $t = 500 \unit{\hour}$
		\item the MTTF
	\end{enumerate}

    \begin{enumerate}[(a)]
        \item 
    \end{enumerate}

\end{question}

%----------------------------------------------------------------------------------------
%	PROBLEM 10
%----------------------------------------------------------------------------------------

\begin{question}
    Let us assume the failure and the repair rates are equal to $\lambda = \num{1.2e-6}$ and $\mu = \num{1.2e-3}$, respectively.
	\begin{enumerate}[(a)]
		\item Calculate the MTTF and MTBF
		\item Calculate the reliability at $t = 8760 \unit{\hour}$
		\item If the annual downtime is $15\unit{\hour}$, what is the steady-state availability?
		\item What is the effect on the availability if we triple the MTTR?
	\end{enumerate}

\end{question}

\begin{enumerate}[(a)]
    \item As the failure and repair rates are constant, we can define MTTF and MTTR as
    \begin{align}
        MTTF &= \frac{1}{\lambda} & 
        MTTR &= \frac{1}{\mu} \label{eq:q10_mttf}
    \end{align}
    Furthermore, assuming that $\text{MNRT} \simeq 0$,
    \begin{align*}
        MTBF = MTTF + MTTR = \frac{1}{\lambda} + \frac{1}{\mu} 
    \end{align*}
    Therefore, using the values provided for $\lambda$ and $\mu$,
    \begin{empheq}[box=\fbox]{align*}
        MTTF &= \frac{1}{\num{1.2e-6}} \\
        &\approx \qty{833333.33}{hours} \\
        MTBF &= \frac{1}{\num{1.2e-6}} + \frac{1}{\num{1.2e-3}} \\
        &\approx \qty{834166.67}{hours}
    \end{empheq} 
    \item As the failure rate is constant, we can define the Reliability $R$ as $$R(t) = e^{-\lambda t}$$
    Therefore, using the values provided for $\lambda$ and $t$,
    \begin{empheq}[box=\fbox]{align*}
        R(t) &= e^{\num{1.2e-6} \times 8760} \\
        &\approx \num{0.9895}
    \end{empheq}
    \item The system downtime ($DT$) can be expressed as 
    \begin{equation}
        DT = (1-A) \times T \label{eq:q10_dt}
    \end{equation}
    Where $T$ is the time period, and $A$ is the steady-state availability. We know $DT = \qty{15}{h}$ refers to the annual downtime, so assuming a year of 365 days
    $$T = 365 \times 24 = \qty{8760}{hours}$$
    Plugging the known variables into equation \eqref{eq:q10_dt}
    \begin{align*}
        15 &= (1-A) \times 8760 \\
        \frac{15}{8760} &= 1-A \\
        A &= 1 - \frac{15}{8760} \\
        &\approx \num{0.9982}
    \end{align*}
    Therefore, the steady-state availability $A$ for a system with 15 hours of downtime is
    \begin{empheq}[box=\fbox]{align*}
        A \approx \num{0.9982}
    \end{empheq} 
    \item We first calculate the original system's availability $A$, given by the formula
    \begin{equation}
        A = \frac{MTTF}{MTTF + MTTR} = \frac{\mu}{\lambda + \mu} \label{eq:q10avail}
    \end{equation}
    From the equations in \eqref{eq:q10_mttf}, and the $MTTF$ we found on item (a), we know that
    \begin{align*}
        MTTF &= \frac{1}{\num{1.2e-6}} & MTTR &= \frac{1}{\num{1.2e-3}}
    \end{align*}
    The original system's availability $A_1$ can then be calculated using \eqref{eq:q10avail}
    \begin{align*}
        A_1 &= \frac{\frac{1}{\num{1.2e-6}}}{\frac{1}{\num{1.2e-6}} + \frac{1}{\num{1.2e-3}}} \\
        &\approx 0.9990
    \end{align*}
    We now triple the MTTR and calculate the new system's availability $A_2$
    \begin{align*}
        A_2 &= \frac{\frac{1}{\num{1.2e-6}}}{\frac{1}{\num{1.2e-6}} + \frac{3}{\num{1.2e-3}}} \\
        &\approx 0.9970
    \end{align*}
    We can then calculate the impact of the new MTTR on the original availability $A_1$ in relative terms
    \begin{empheq}[box=\fbox]{align*}
    1 - \frac{A_2}{A_1} = \num{0.002} = 0.2\%
    \end{empheq}
    Therefore, tripling the MTTR brought down the availability $A_1$ of the system by 0.2\%

\end{enumerate}

%----------------------------------------------------------------------------------------

\end{document}
