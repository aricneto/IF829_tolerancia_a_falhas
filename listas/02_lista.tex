%%%%%%%%%%%%%%%%%%%%%%%%%%%%%%%%%%%%%%%%%
% Lachaise Assignment
% LaTeX Template
% Version 1.0 (26/6/2018)
%
% This template originates from:
% http://www.LaTeXTemplates.com
%
% Authors:
% Marion Lachaise & François Févotte
% Vel (vel@LaTeXTemplates.com)
%
% License:
% CC BY-NC-SA 3.0 (http://creativecommons.org/licenses/by-nc-sa/3.0/)
% 
%%%%%%%%%%%%%%%%%%%%%%%%%%%%%%%%%%%%%%%%%

%----------------------------------------------------------------------------------------
%	PACKAGES AND OTHER DOCUMENT CONFIGURATIONS
%----------------------------------------------------------------------------------------

\documentclass{article}
% \usepackage[brazil]{babel}
\usepackage{graphicx}
\graphicspath{ {./images/} }

\input{structure.tex} % Include the file specifying the document structure and custom commands

%----------------------------------------------------------------------------------------
%	ASSIGNMENT INFORMATION
%----------------------------------------------------------------------------------------

\title{Critical Systems Evaluation} % Title of the assignment

\author{First Homework List\\ Equipe: \texttt{aasb2, acn2, pvoa, vbmp}} % Author name and email address

\date{CIn, UFPE --- \today} % University, school and/or department name(s) and a date

%----------------------------------------------------------------------------------------

\begin{document}

\maketitle % Print the title

%----------------------------------------------------------------------------------------
%	PROBLEM 5
%----------------------------------------------------------------------------------------
\setcounter{Question}{4}
\begin{question}
    Table 3 summarizes the reliability data test of eighty servers $(n = 80)$ during $13500h$. Inspections were carried out every week $(500h)$. The whole observation was finished at $13500h$. Table 3 shows the time instants that represent the end of each of the twenty-seven intervals $(Column \ t_i)$, the number of failed servers in the respective interval $(Column \ r_i)$, and the number of censored computer in the interval $(Column \ c_i)$. Adopted a non-parametric method to draw the graphs
    \begin{enumerate}[label=(\alph*)]
        \item $\hat{R}(t) \times t$
        \item $\hat{F}(t) \times t$
        \item Estimate $\hat{R}(8500h)$
    \end{enumerate}
\end{question}

\begin{enumerate}[label=(\alph*)]
    \item We'll use the non-parametric Kaplan-Meier estimator to calculate the desired functions. This method is appropriate for this dataset since it can correctly account for right-censored data. For the survival function $R(t)$, we have
    \[\hat{R}(t) = \prod_{i: t_i<t} \left(1- \frac{d_i}{n_i}\right)\]
    Where $d_i$ is the number of failures at time $t_i$, and $n_i$ is the survivors up to time $t_i$. That is, the total number of servers minus the sum of all failures and censored data before $t_i$.
    We used the provided Life Time Data Analysis spreadsheet to compute the results for the Kaplan-Meier estimator.

    \begin{figure}[H]
        \centering
        \includegraphics[width=0.5\linewidth]{q3_sheet.png}
        \caption{Kaplan-Meier spreadsheet results}
        \label{fig:q3_km1}
    \end{figure}

    With the resulting values, we can plot the curve for the $\hat{R}(t)$ and $\hat{F}(t)$ functions

    \begin{figure}[H]
        \centering
        \includegraphics[width=0.8\linewidth]{q3_rt.png}
        \caption{$\hat{R}(t) \times t$}
        \label{fig:q3_km_rt}
    \end{figure}

    We can see that the reliability is fairly stable, being above 0.65 after 13500h
    \item We know that $\hat{F}(t) = 1 - \hat{R}(t)$, so we can take the complement of the values from $\hat{R}(t)$ to plot $\hat{F}(t)$

    \begin{figure}[H]
        \centering
        \includegraphics[width=0.8\linewidth]{q3_ft.png}
        \caption{$\hat{F}(t) \times t$}
        \label{fig:q3_km_ft}
    \end{figure}

    \item As $t=8500$ is actually one of our data points, we can use the value we estimated for $t=8500$ directly for our reliability estimate, thus, \fbox{$\hat{R}(8500) \approx 0.822$}. Alternatively, we can run a regression on our data points, and estimate the reliability at that point using that regression. In case of a linear regression, we get $\hat{R}(8500) \approx 0.861$.
\end{enumerate}

%----------------------------------------------------------------------------------------
%	PROBLEM 7
%----------------------------------------------------------------------------------------

\setcounter{Question}{6}
\begin{question}
    What are the pros and cons of the following parametric reliability data analysis:
    \begin{enumerate}[label=(\alph*)]
        \item graphical methods

        \item methods of moments, and
        \item  maximum likelihood estimation methods.
    \end{enumerate}
\end{question}

\begin{enumerate}[label=(\alph*)]
    \item \textbf{Graphical methods:}

          Pros:
          \begin{itemize}
              \item Easy and simple to use
              \item Can be visually checked
          \end{itemize}
          Cons:
          \begin{itemize}
              \item They are not precise
          \end{itemize}

    \item \textbf{Methods of moments:}

          Pros:
          \begin{itemize}
              \item Easy to implement
              \item Flexibility
              \item  Does not depend on distibution assumption
          \end{itemize}
          Cons:
          \begin{itemize}
              \item Sensitive to outliers
              \item Might ignore some distribution aspects
          \end{itemize}

    \item \textbf{Likelihood estimation methods:}

          Pros:
          \begin{itemize}
              \item Less Likely to Be Biased
          \end{itemize}
          Cons:
          \begin{itemize}
              \item Not precise for small number of failures
              \item It might be hard to calculate in some cases
          \end{itemize}
\end{enumerate}

%----------------------------------------------------------------------------------------
%	PROBLEM 8
%----------------------------------------------------------------------------------------

\begin{question}
    Assume the reliability experiment in which sixty servers (n = 60) were observed in an accelerated reliability test. The times each server failed were recorded in Table \ref{table:4} (complete data).
    \begin{enumerate}[label=(\alph*)]
        \item Conduct an exploratory analysis and choose a candidate probability distribution representing the data set.
        \item  Obtain a point estimate of distribution parameters and conduct goodness of fitting. If the candidate distribution does not fit the data set, choose another one until finding a
              suitable distribution.
        \item  Obtain a point estimate of the $MTTF$
        \item Estimate the $MTTF$ confidence interval, adopting 5\% of significance using a semi-parametric bootstrap.
        \item  Estimate the $R(t)$, $F(t)$, $f(t)$ and $\lambda(t)$ at $1000 h$
    \end{enumerate}

\end{question}

\begin{table}[H]
    \centering
    \begin{tabular}{|*{12}{c|}}
        \hline
        $i$ & $t_i$ (h) & $i$ & $t_i$ (h) & $i$ & $t_i$ (h) & $i$ & $t_i$ (h) & $i$ & $t_i$ (h) & $i$ & $t_i$ (h) \\
        \hline
        1   & 765.32    & 11  & 1822.54   & 21  & 515.21    & 31  & 4569.68   & 41  & 408.16    & 51  & 338.18    \\
        2   & 736.36    & 12  & 2026.03   & 22  & 5400.53   & 32  & 4104.01   & 42  & 1509.31   & 52  & 197.89    \\
        3   & 4688.02   & 13  & 211.91    & 23  & 1460.71   & 33  & 728.45    & 43  & 434.26    & 53  & 584.37    \\
        4   & 2181.27   & 14  & 1109.13   & 24  & 1672.60   & 34  & 1188.55   & 44  & 2005.96   & 54  & 86.76     \\
        5   & 4085.90   & 15  & 1189.89   & 25  & 640.21    & 35  & 640.83    & 45  & 924.12    & 55  & 2204.17   \\
        6   & 9044.38   & 16  & 1348.37   & 26  & 711.69    & 36  & 5613.10   & 46  & 1510.17   & 56  & 6591.52   \\
        7   & 3573.45   & 17  & 389.56    & 27  & 190.62    & 37  & 247.44    & 47  & 2548.00   & 57  & 6023.71   \\
        8   & 7201.78   & 18  & 3563.30   & 28  & 4266.00   & 38  & 1647.28   & 48  & 157.32    & 58  & 398.62    \\
        9   & 1237.32   & 19  & 2082.53   & 29  & 1087.26   & 39  & 500.76    & 49  & 399.17    & 59  & 4064.59   \\
        10  & 740.13    & 20  & 3322.64   & 30  & 489.08    & 40  & 3510.87   & 50  & 2020.38   & 60  & 2467.03   \\
        \hline
    \end{tabular}
    \label{table:4}
\end{table}
\begin{enumerate}[label=(\alph*)]
    \item The result of the exploratory analysis was the following:
          \begin{flushleft}
              Sample Size, $n$: 60 \\
              Mean: 2089.64000 \\
              Median: 1404.54000 \\
              Midrange: 4565.57000 \\
              RMS: 2896.88557 \\
              Variance, $s^2$: 4093576.93357 \\
              Standard Deviation, $s$: 2023.25899 \\
              Mean Absolute Deviation: 1574.37167 \\
              Range: 8957.62000 \\
              Coefficient of Variation: 96.82333\%

              Minimum: 86.76 \\
              1st Quartile: 549.79000 \\
              2nd Quartile: 1404.54000 \\
              3rd Quartile: 3416.75500 \\
              Maximum: 9044.38

              Sum: 125378.40000 \\
              Sum of Squares: 503516758.85680

              95\% CI for the Mean:
              $1566.97680 < \text{mean} < 2612.30320$

              95\% CI for the Standard Deviation:
              $1714.98373 < \text{SD} < 2467.69075$

              95\% CI for the Variance:
              $2941169.20755 < \text{VAR} < 6089497.63452$
          \end{flushleft}

          \begin{figure}[H]
              \centering
              \includegraphics[width=0.8\linewidth]{statdisk_boxplot_q8.png}
              \caption{Boxplot}
              \label{fig:boxplot_q8}
          \end{figure}

          \begin{figure}[H]
              \centering
              \includegraphics[width=0.8\linewidth]{statdisk_histogram_q8.png}
              \caption{Histogram}
              \label{fig:histogram_q8}
          \end{figure}

          \begin{figure}[H]
              \centering
              \includegraphics[width=0.8\linewidth]{statdisk_normal_quantile_plot_q8.png}
              \caption{Normal Quantile Plot}
              \label{fig:normalquantile_q8}
          \end{figure}

          According to \ref{fig:normalquantile_q8} we can already discard the Normal distribution. Based on the Box Plot in \ref{fig:boxplot_q8} and the Histogram in \ref{fig:histogram_q8} a good guess would be the exponential distribution.

    \item Before conducting the the goodness of fitting it is necessary to estimate the distribution parameter $\lambda$ for $f(t)=\lambda e^{-\lambda t}$, to do it, it can be used the maximum likelihood method, it can be done substituting the obtained values for the distribution on the following equation:

          \begin{equation}
              \lambda=\frac{n}{\sum\limits_{i=1}^{n}t_i}
          \end{equation}


          where $n=60$ and $\sum\limits_{i=1}^{n}t_i=125378.40$. so the value for $\lambda$ is:
          \begin{empheq}[box=\fbox]{align*}
              \lambda\approx 0.0004786
          \end{empheq}

          The Kolmogorov Smirnov can be used to check if the distribution fits:
          \begin{figure}[H]
              \centering
              \includegraphics[width=1\linewidth]{ks_q8.png}
              \caption{Kolmogorov Smirnov Method}
              \label{fig:graphical2_q8}
          \end{figure}

          As the Kolmogorov Smirnov test failed to reject the null hypothesis so it is safe to assume
          the exponential distribution may fit the data.

          \begin{figure}[H]
              \centering
              \includegraphics[width=1\linewidth]{easyfit_q8.png}
              \caption{Easy Fit}
              \label{fig:easyfit_q8}
          \end{figure}

          Using easy fit to check if the exponential method it can be seem that the exponential distribution ranks pretty high among the possible distributions that can fit the data using the Kolmogrov Smirnov, Anderson Darling and Chi-Squared methods, There are better distributions according to Easy Fit for the data, but it will be used the exponential distribution because it was the one first assumed and the null hypothesis was not rejected.

    \item An estimation for $MTTF$ can be obtained by simply taking the mean from the failure time on the table, or by calculating $1/\lambda$. The result obtained after doing this is:

          \begin{empheq}[box=\fbox]{align*}
              MTTF = \frac{\sum\limits_{i=1}^{n}t_i}{n} = 2089.64
          \end{empheq}


    \item Using the following code in python to apply the semi-parametric bootstrap to the data:

          \lstset{
              basicstyle=\ttfamily,
              numbers=left,
              numberstyle=\tiny\color{gray},
              breaklines=true,
              keywordstyle=\color{blue},
              commentstyle=\color{green!40!black},
              stringstyle=\color{orange},
              frame=single,
              language=Python
          }


          \begin{lstlisting}
    import numpy as np
    
    data = [765.32, 1822.54, 515.21, 4569.68, 408.16, 338.18, 736.36, 2026.03, 5400.53, 4104.01, 1509.31, 197.89, 4688.02, 211.91, 1460.71, 728.45, 434.26, 584.37, 2181.27, 1109.13, 1672.6, 1188.55, 2005.96, 86.76, 4085.9, 1189.89, 640.21, 640.83, 924.12, 2204.17, 9044.38, 1348.37, 711.69, 5613.1, 1510.17, 6591.52, 3573.45, 389.56, 190.62, 247.44, 2548.0, 6023.71, 7201.78, 3563.3, 4266.0, 1647.28, 157.32, 398.62, 1237.32, 2082.53, 1087.26, 500.76, 399.17, 4064.59, 740.13, 3322.64, 489.08, 3510.87, 2020.38, 2467.03]
    data.sort()
    data = np.array(data)
    iterations = 1000
    MTTF = np.mean(data)
    mttfs = []
    
    for _ in range(iterations):
      # non_parametric_data = np.random.choice(data,size=len(data),replace=True)
      parametric_data = np.random.exponential(scale=MTTF, size=len(data))
      mttfs.append(np.mean(parametric_data))
    
    mttfs.sort()
    confidence_interval = np.percentile(mttfs, [2.5, 97.5])
    print(confidence_interval)
    \end{lstlisting}

          the confidence interval for $MTTF$ with 5\% of significance is :


          \begin{empheq}[box=\fbox]{align*}
              [1624.6652h,~2631.3845h]
          \end{empheq}

    \item
    For an exponential distribution the Reliability function is:
    \begin{equation}
        R(t)=e^{-\lambda t}=e{-0.0004786 t}
    \end{equation}

    so for $t=1000h$ then:
    \begin{empheq}[box=\fbox]{align*}
        R(t=1000)=e^{-0.0004786 \cdot 1000} \approx 0.619
    \end{empheq}

    For an exponential distribution the Failure function is:
    \begin{equation}
        F(t)=1-R(t)
    \end{equation}

    so for $t=1000h$ then:
    \begin{empheq}[box=\fbox]{align*}
        F(t=1000)=1-0.619 \approx 0.380
    \end{empheq}

    For an exponential distribution the density function is:
    \begin{equation}
        f(t)=\lambda e^{-\lambda t}=0.0004786\cdot e^{-0.0004786 t}
    \end{equation}

    So for $t=1000h$ then:
    \begin{empheq}[box=\fbox]{align*}
        f(t=1000)=0.0004786\cdot e^{-0.0004786 \cdot 1000} \approx 0.000297
    \end{empheq}
    For an exponential distribution the hazard function is constant:

    \begin{equation}
        \lambda(t)=\lambda=0.0004786
    \end{equation}

    So for $t=1000h$ then:
    \begin{empheq}[box=\fbox]{align*}
        \lambda(t=1000)=\lambda=0.0004786
    \end{empheq}
\end{enumerate}

%----------------------------------------------------------------------------------------
%	PROBLEM 9
%----------------------------------------------------------------------------------------
\setcounter{Question}{8}
\begin{question}
    Consider the reliability experiment in which sixty servers \((n = 60)\) were observed in an accelerated reliability test. The times each server failed were recorded in Table 4 (complete data). Using a non-parametric method, calculate
    \begin{enumerate}[label=(\alph*)]
        \item \(\hat{R}(t) \text{ at } t = 1000h\)
        \item \(\hat{F}(t) \text{ at } t = 1000h\)
        \item \(\hat{f}(t) \text{ at } t = 1000h\)
        \item \(\hat{\lambda}(t) \text{ at } t = 1000h\)
    \end{enumerate}
    Draw:
    \begin{enumerate}[label=(\alph*), start=5]
        \item \(\hat{R}(t_i) \times t_i\)
        \item \(\hat{F}(t_i) \times t_i\)
        \item \(\hat{f}(t_i) \times t_i\)
        \item \(\hat{\lambda}(t) \times t\)
    \end{enumerate}
    And:
    \begin{enumerate}[label=(\alph*), start=9]
        \item \(\widehat{MTTF}\)
        \item Estimate the confidence interval using non-parametric bootstrap.
        \item Compare the results with Exercise 8 and comment.
    \end{enumerate}
\end{question}

As we have ungrouped and complete reliability data, we can use a simple method to estimate each function. All the data was gathered in an Excel sheet, and the following equations were used for non-parametrically estimating the desired functions:
    \begin{align*}
        \hat{R}(t_i) &= \frac{n+1-i}{n+1}    &   \hat{F}(t_i) &= \frac{i}{n+1} \\
        \hat{f}(t_i) &= \frac{1}{(n+1)(t_i - t_{i - 1})}   &   \hat{\lambda}(t_i) &= \frac{1}{(n+1-i)(t_i - t_{i-1})}
    \end{align*}
    Where $(n - i)$ is the number of surviving servers at time $t_i$. Some of the calculated results can be seen in the following table:

    \begin{table}[H]
        \centering
        \caption{Estimated functions}
        \label{tab:my-table}
        \begin{tabular}{|c|c|cccc|}
            \hline
            $i$ & $t_i$ & $R(t_i)$ & $F(t_i)$ & $f(t_i)$ & $l(t_i)$ \\
            \hline
            1  & 86.76  & 0.9836 & 0.0164 & 0.0000 & 0.0000 \\
            2  & 157.32 & 0.9672 & 0.0328 & 0.0002 & 0.0002 \\
            3  & 190.62 & 0.9508 & 0.0492 & 0.0005 & 0.0005 \\
            4  & 197.89 & 0.9344 & 0.0656 & 0.0023 & 0.0024 \\
            5  & 211.91 & 0.9180 & 0.0820 & 0.0012 & 0.0013 \\
            $\cdots$ & $\cdots$ & $\cdots$ & $\cdots$ & $\cdots$ & $\cdots$ \\
            20 & 728.45 & 0.6721 & 0.3279 & 0.0010 & 0.0015 \\
            21 & 736.36 & 0.6557 & 0.3443 & 0.0021 & 0.0032 \\
            22 & 740.13 & 0.6393 & 0.3607 & 0.0043 & 0.0068 \\
            23 & 765.32 & 0.6230 & 0.3770 & 0.0007 & 0.0010 \\
            24 & 924.12 & 0.6066 & 0.3934 & 0.0001 & 0.0002 \\
            25 & 1087.26 & 0.5902 & 0.4098 & 0.0001 & 0.0002 \\
            26 & 1109.13 & 0.5738 & 0.4262 & 0.0007 & 0.0013 \\
            27 & 1188.55 & 0.5574 & 0.4426 & 0.0002 & 0.0004 \\
            28 & 1189.89 & 0.5410 & 0.4590 & 0.0122 & 0.0226 \\
            $\cdots$ & $\cdots$ & $\cdots$ & $\cdots$ & $\cdots$ & $\cdots$ \\
            55 & 5400.53 & 0.0984 & 0.9016 & 0.0000 & 0.0002 \\
            56 & 5613.1 & 0.0820 & 0.9180 & 0.0001 & 0.0009 \\
            57 & 6023.71 & 0.0656 & 0.9344 & 0.0000 & 0.0006 \\
            58 & 6591.52 & 0.0492 & 0.9508 & 0.0000 & 0.0006 \\
            59 & 7201.78 & 0.0328 & 0.9672 & 0.0000 & 0.0008 \\
            60 & 9044.38 & 0.0164 & 0.9836 & 0.0000 & 0.0005 \\
            \hline
        \end{tabular}
    \end{table}

    With the calculated data, we can then use the following formula to calculate the values for each function at $t = 1000$:
    \[\hat{R} = (t_j) = a \times t_j + b\]
    Where $t_i < t_j < t_{i+1}$, such that $t_j$ is the desired $t$, and $t_i, t_{i+1}$ are two recorded failure times, and
    \[a = \frac{\hat{R}(t_{i+1})- \hat{R}(t_i)}{t_{i+1} - t_i}\]
    \[b = \hat{R}(t_i) - a \times t_i\]

    \begin{enumerate}[label=(\alph*)]
        \item We choose $i = 24$ for our estimation, as $924.12 < 1000 < 1087.26$. Therefore, we have:
        \begin{gather*}
            a = \frac{0.5902 - 0.6066}{1087.26 - 924.12} = \num{-1,0052e-4} \\
            b = 0.6066 - (\num{-1,0052e-4} \times 924.12) = 0,6994 \\
            \hat{R}(1000) = (\num{-1,0052e-4} \times 1000) + 0,6994 \\
            \fbox{$\hat{R}(1000) = 0,5988$}
        \end{gather*}
        \item Using a similar approach, we can calculate the rest of the functions at $t = 1000$
        \[\fbox{$\hat{F}(1000) = 0.4010$}\]
        \item \[\fbox{$\hat{f}(1000) = 0.0001$}\]
        \item \[\fbox{$\hat{\lambda}(1000) = 0.0001$}\]
        \item $R(t)$
        \begin{figure}[H]
            \centering
            \includegraphics[width=0.8\linewidth]{q9_rt.png}
            \caption{$\hat{R}(t) \times t$}
            \label{fig:q9_rt}
        \end{figure}
        \item $F(t)$
        \begin{figure}[H]
            \centering
            \includegraphics[width=0.8\linewidth]{q9_FFt.png}
            \caption{$\hat{F}(t) \times t$}
            \label{fig:q9_Ft}
        \end{figure}
        \item $f(t)$
        \begin{figure}[H]
            \centering
            \includegraphics[width=0.8\linewidth]{q9_ft.png}
            \caption{$\hat{F}(t) \times t$}
            \label{fig:q9_ft}
        \end{figure}
        \item $\lambda(t)$
        \begin{figure}[H]
            \centering
            \includegraphics[width=0.8\linewidth]{q9Lt.png}
            \caption{$\hat{F}(t) \times t$}
            \label{fig:q9_lt}
        \end{figure}
        \item The MTTF is estimated by
        \[\widehat{MTTF} = \frac{\sum_{i=1}^{n}t_i}{n}\]
        Computing this, we get:
        \[\fbox{$MTTF = 2089.64h$}\]
        \item Assuming a confidence level of 95\%, upon calculating the confidence interval using non-parametric bootstrap, we get \fbox{MTTF $\in (1599.03h,~ 2609.37h)$}
        \item The results compared to the ones found in Exercise 8 are very close, which demonstrates that the semi-parametric approach chosen in the previous exercise seems to accurately represent the original dataset.
    \end{enumerate}
    
\end{document}
