%%%%%%%%%%%%%%%%%%%%%%%%%%%%%%%%%%%%%%%%%
% Lachaise Assignment
% LaTeX Template
% Version 1.0 (26/6/2018)
%
% This template originates from:
% http://www.LaTeXTemplates.com
%
% Authors:
% Marion Lachaise & François Févotte
% Vel (vel@LaTeXTemplates.com)
%
% License:
% CC BY-NC-SA 3.0 (http://creativecommons.org/licenses/by-nc-sa/3.0/)
% 
%%%%%%%%%%%%%%%%%%%%%%%%%%%%%%%%%%%%%%%%%

%----------------------------------------------------------------------------------------
%	PACKAGES AND OTHER DOCUMENT CONFIGURATIONS
%----------------------------------------------------------------------------------------

\documentclass{article}
% \usepackage[brazil]{babel}
\usepackage{graphicx}
\graphicspath{ {./images/} }

\input{structure.tex} % Include the file specifying the document structure and custom commands

%----------------------------------------------------------------------------------------
%	ASSIGNMENT INFORMATION
%----------------------------------------------------------------------------------------

\title{Critical Systems Evaluation} % Title of the assignment

\author{First Homework List\\ Equipe: \texttt{aasb2, acn2, pvoa, vbmp}} % Author name and email address

\date{CIn, UFPE --- \today} % University, school and/or department name(s) and a date

%----------------------------------------------------------------------------------------

\begin{document}

\maketitle % Print the title

%----------------------------------------------------------------------------------------
%	PROBLEM 5
%----------------------------------------------------------------------------------------
\setcounter{Question}{4}
\begin{question}
    Table 3 summarizes the reliability data test of eighty servers $(n = 80)$ during $13500h$. Inspections were carried out every week $(500h)$. The whole observation was finished at $13500h$. Table 3 shows the time instants that represent the end of each of the twenty-seven intervals $(Column \ t_i)$, the number of failed servers in the respective interval $(Column \ r_i)$, and the number of censored computer in the interval $(Column \ c_i)$. Adopted a nonparametric method to draw the graphs
    \begin{enumerate}[label=(\alph*)]
        \item $\hat{R}(t) \times t$
        \item $\hat{F}(t) \times t$
        \item Estimate $\hat{R}(8500h)$
    \end{enumerate}
\end{question}

\begin{enumerate}[label=(\alph*)]
    \item We'll use the non-parametric Kaplan-Meier estimator to calculate the desired functions. This method is appropriate for this dataset since it can correctly account for right-censored data. For the survival function $R(t)$, we have
    \[\hat{R}(t) = \prod_{i: t_i<t} \left(1- \frac{d_i}{n_i}\right)\]
    Where $d_i$ is the number of failures at time $t_i$, and $n_i$ is the survivors up to time $t_i$. That is, the total number of servers minus the sum of all failures and censored data before $t_i$.
    We used the provided Life Time Data Analysis spreadsheet to compute the results for the Kaplan-Meier estimator.

    \begin{figure}[H]
        \centering
        \includegraphics[width=0.5\linewidth]{q3_sheet.png}
        \caption{Kaplan-Meier spreadsheet results}
        \label{fig:q3_km1}
    \end{figure}

    With the resulting values, we can plot the curve for the $\hat{R}(t)$ and $\hat{F}(t)$ functions

    \begin{figure}[H]
        \centering
        \includegraphics[width=0.8\linewidth]{q3_rt.png}
        \caption{$\hat{R}(t) \times t$}
        \label{fig:q3_km_rt}
    \end{figure}

    We can see that the reliability is fairly stable, being above 0.65 after 13500h
    \item We know that $\hat{F}(t) = 1 - \hat{R}(t)$, so we can take the complement of the values from $\hat{R}(t)$ to plot $\hat{F}(t)$

    \begin{figure}[H]
        \centering
        \includegraphics[width=0.8\linewidth]{q3_ft.png}
        \caption{$\hat{F}(t) \times t$}
        \label{fig:q3_km_ft}
    \end{figure}

    \item As $t=8500$ is actually one of our data points, we can use the value we estimated for $t=8500$ directly for our reliability estimate, thus, \fbox{$\hat{R}(8500) \approx 0.822$}. Alternatively, we can run a regression on our data points, and estimate the reliability at that point using that regression. In case of a linear regression, we get $\hat{R}(8500) \approx 0.861$.
\end{enumerate}

%----------------------------------------------------------------------------------------
%	PROBLEM 9
%----------------------------------------------------------------------------------------
\setcounter{Question}{8}
\begin{question}
    Consider the reliability experiment in which sixty servers \((n = 60)\) were observed in an accelerated reliability test. The times each server failed were recorded in Table 4 (complete data). Using a non-parametric method, calculate
    \begin{enumerate}[label=(\alph*)]
        \item \(\hat{R}(t) \text{ at } t = 1000h\)
        \item \(\hat{F}(t) \text{ at } t = 1000h\)
        \item \(\hat{f}(t) \text{ at } t = 1000h\)
        \item \(\hat{\lambda}(t) \text{ at } t = 1000h\)
    \end{enumerate}
    Draw:
    \begin{enumerate}[label=(\alph*), start=5]
        \item \(\hat{R}(t_i) \times t_i\)
        \item \(\hat{f}(t_i) \times t_i\)
        \item \(\hat{\lambda}(t) \times t\)
    \end{enumerate}
    And:
    \begin{enumerate}[label=(\alph*), start=8]
        \item \(\widehat{MTTF}\)
        \item Estimate the confidence interval using nonparametric bootstrap.
        \item Compare the results with Exercise 8 and comment.
    \end{enumerate}
\end{question}

As we have ungrouped and complete reliability data, we can use a simple method to estimate each function. All the data was gathered in an Excel sheet, and the following equations were used for non-parametrically estimating the desired functions:
    \begin{align*}
        \hat{R}(t_i) &= \frac{n+1-i}{n+1}    &   \hat{F}(t_i) &= \frac{i}{n+1} \\
        \hat{f}(t_i) &= \frac{1}{(n+1)(t_i - t_{i - 1})}   &   \hat{\lambda}(t_i) &= \frac{1}{(n+1-i)(t_i - t_{i-1})}
    \end{align*}
    Where $(n - i)$ is the number of surviving servers at time $t_i$. Some of the calculated results can be seen in the following table:

    \begin{table}[H]
        \centering
        \caption{Estimated functions}
        \label{tab:my-table}
        \begin{tabular}{|c|c|cccc|}
            \hline
            $i$ & $t_i$ & $R(t_i)$ & $F(t_i)$ & $f(t_i)$ & $l(t_i)$ \\
            \hline
            1  & 86.76  & 0.9836 & 0.0164 & 0.0000 & 0.0000 \\
            2  & 157.32 & 0.9672 & 0.0328 & 0.0002 & 0.0002 \\
            3  & 190.62 & 0.9508 & 0.0492 & 0.0005 & 0.0005 \\
            4  & 197.89 & 0.9344 & 0.0656 & 0.0023 & 0.0024 \\
            5  & 211.91 & 0.9180 & 0.0820 & 0.0012 & 0.0013 \\
            $\cdots$ & $\cdots$ & $\cdots$ & $\cdots$ & $\cdots$ & $\cdots$ \\
            20 & 728.45 & 0.6721 & 0.3279 & 0.0010 & 0.0015 \\
            21 & 736.36 & 0.6557 & 0.3443 & 0.0021 & 0.0032 \\
            22 & 740.13 & 0.6393 & 0.3607 & 0.0043 & 0.0068 \\
            23 & 765.32 & 0.6230 & 0.3770 & 0.0007 & 0.0010 \\
            24 & 924.12 & 0.6066 & 0.3934 & 0.0001 & 0.0002 \\
            25 & 1087.26 & 0.5902 & 0.4098 & 0.0001 & 0.0002 \\
            26 & 1109.13 & 0.5738 & 0.4262 & 0.0007 & 0.0013 \\
            27 & 1188.55 & 0.5574 & 0.4426 & 0.0002 & 0.0004 \\
            28 & 1189.89 & 0.5410 & 0.4590 & 0.0122 & 0.0226 \\
            $\cdots$ & $\cdots$ & $\cdots$ & $\cdots$ & $\cdots$ & $\cdots$ \\
            55 & 5400.53 & 0.0984 & 0.9016 & 0.0000 & 0.0002 \\
            56 & 5613.1 & 0.0820 & 0.9180 & 0.0001 & 0.0009 \\
            57 & 6023.71 & 0.0656 & 0.9344 & 0.0000 & 0.0006 \\
            58 & 6591.52 & 0.0492 & 0.9508 & 0.0000 & 0.0006 \\
            59 & 7201.78 & 0.0328 & 0.9672 & 0.0000 & 0.0008 \\
            60 & 9044.38 & 0.0164 & 0.9836 & 0.0000 & 0.0005 \\
            \hline
        \end{tabular}
    \end{table}

    With the calculated data, we can then use the following formula to calculate the values for each function at $t = 1000$:
    \[\hat{R} = (t_j) = a \times t_j + b\]
    Where $t_i < t_j < t_{i+1}$, such that $t_j$ is the desired $t$, and $t_i, t_{i+1}$ are two recorded failure times, and
    \[a = \frac{\hat{R}(t_{i+1})- \hat{R}(t_i)}{t_{i+1} - t_i}\]
    \[b = \hat{R}(t_i) - a \times t_i\]

    \begin{enumerate}[label=(\alph*)]
        \item We choose $i = 24$ for our estimation, as $924.12 < 1000 < 1087.26$. Therefore, we have:
        \begin{gather*}
            a = \frac{0.5902 - 0.6066}{1087.26 - 924.12} = \num{-1,0052e-4} \\
            b = 0.6066 - (\num{-1,0052e-4} \times 924.12) = 0,6994 \\
            \hat{R}(1000) = (\num{-1,0052e-4} \times 1000) + 0,6994 \\
            \fbox{$\hat{R}(1000) = 0,5988$}
        \end{gather*}
        \item Using a similar approach, we can calculate the rest of the functions at $t = 1000$
        \[\fbox{$\hat{F}(1000) = 0.4010$}\]
        \item \[\fbox{$\hat{f}(1000) = 0.0001$}\]
        \item \[\fbox{$\hat{\lambda}(1000) = 0.0001$}\]
        \item $R(t)$
        \begin{figure}[H]
            \centering
            \includegraphics[width=0.8\linewidth]{q9_rt.png}
            \caption{$\hat{R}(t) \times t$}
            \label{fig:q9_rt}
        \end{figure}
        \item $F(t)$
        \begin{figure}[H]
            \centering
            \includegraphics[width=0.8\linewidth]{q9_FFt.png}
            \caption{$\hat{F}(t) \times t$}
            \label{fig:q9_Ft}
        \end{figure}
        \item $f(t)$
        \begin{figure}[H]
            \centering
            \includegraphics[width=0.8\linewidth]{q9_ft.png}
            \caption{$\hat{F}(t) \times t$}
            \label{fig:q9_ft}
        \end{figure}
        \item $\lambda(t)$
        \begin{figure}[H]
            \centering
            \includegraphics[width=0.8\linewidth]{q9Lt.png}
            \caption{$\hat{F}(t) \times t$}
            \label{fig:q9_lt}
        \end{figure}
    \end{enumerate}
    
\end{document}
